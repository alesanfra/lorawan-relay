\chapter*{Abstract}

LPWANs have recently arisen as game changer in the field of the Internet of Things thanks to the wide coverage area, low cost of adoption and maintenance, and very low power consumption. Among the many incompatible technologies present on the market, LoRa seems to be the most promising one, combining good performance to an open specification of its MAC layer, called LoRaWAN. Because of these reasons LoRa immediately attracted the attention of both the scientific community and the industrial world, making it one of the most widely used LPWAN technology in the world. 

The aim of this work is to perform a deep and complete evaluation of the LoRa technology, exploring all the possibilities offered by the numerous parameters on which is possible to operate. To achieve a complete control of the network a brand new platform independent server infrastructure was developed from scratch, and it was designed to be at the same time lightweight and flexible for the experimental needs.

The first phase of experiments was conducted by exploring all possible, but reasonable, combination of data rate, transmission power and forward error correction levels. The analysis of the results leaded to the design of an extension of the LoRaWAN protocol to enable relay based communication. Finally, a new set of experiments was performed in order to prove the performance improvement compared to the standard LoRaWAN solution.

